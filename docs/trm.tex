\documentclass[10pt]{amsart}

\title{Introduction to Photorealistic Path Tracing}
\author{Arden Rasmussen}
\date{\today}

\begin{document}

\begin{abstract}
    This paper provides a brief introduction into advanced computer graphics
    techniques, such as path tracing, photorealistic rendering, BSDF, and
    volumetric rendering.
\end{abstract}

\maketitle

\section{Introduction}\label{sec:introduction}

Photorealistic based rendering is a common method that is used in modern
computer graphics. It used to be limited to offline rendering, but has recently
been improved to be usable in realtime rendering. This paper will primarily
focus on the application of photorealistic rendering in offline systems. This
means that the rendering process can be significantly slower, and thus allows
the opportunity to produce more realistic images.

The fact that the image rendering is not being done in real time does not mean
that it is not important to optimize the code. In fact the methods that are
used for offline photorealistic rendering are significantly slower than the
methods that are used for realtime rendering. However, they produce images that
represent reality much more accurately.

This paper will discuss the following concepts:
\begin{itemize}
    \item Path Tracing
    \item Ray Marching
    \item Signed Distance Functions
    \item Bidirectional Scattering Distribution Functions
    \item Volumetric Rendering
    \item Parallel Processing
    \item GPU Acceleration
    \item Neural Network Denoising
\end{itemize}

\section{Ray Marching}\label{sec:ray_marching}
\section{Path Tracing}\label{sec:path_tracing}
\section{BSDF}\label{sec:bsdf}
\section{Volumetric Rendering}\label{sec:volumetric_rendering}

\end{document}
